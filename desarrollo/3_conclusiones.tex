\chapter{Conclusiones, Resultados y Recomendaciones}

El desarrollo de las pasantías permitió conocer diferentes aspectos relacionados con el ámbito empresarial y el desarrollo de proyectos de manera efectiva en éste ambiente. Resalta el benéfico del contacto directo y constante con la dirección general del proyecto, ya que esto no sólo beneficia la calidad del producto generado, sino que también permite aportar directamente al modelo de negocio y su crecimiento. 

Ha sido un reto constante para Empresas TD ofrecer soluciones tecnológicas e innovadoras orientadas a mejorar la calidad del comercio electrónico en Venezuela. Este proyecto de pasantía representa la primera aplicación móvil de desarrollo nativo que se realiza en la empresa, por lo que significó la apertura a una nueva forma de acercamiento a sus clientes y la garantía de mantenerse al día con las nuevas tendencias mundiales.

Las metodologías aplicadas al igual que otros procesos de desarrollo ágil dan prioridad a  satisfacer al cliente mediante la entrega continua de funcionalidad y la capacidad de que los requisitos cambien, incluso en etapas tardías del desarrollo. Adaptarse a esta estructura de trabajo tan dinámica es la única manera de vivir al día con los avances tecnológicos y las tendencias globales. Las empresas que logran llevar estos dos elementos de la mano, son las que tienen mayor éxito en nuestros días. 

Este dinamismo al que estuvo sujeto el proyecto afecto también muy positivamente la experiencia y el aprendizaje adquirido. Si bien el objetivo principal de la pasantía era el desarrollo de una aplicación Android, tambien se pudo interactuar con otras tecnologías como Drupal y Apache Solr, lo que represento la adquisición de un conocimiento valioso que no estaba planificado.

El proyecto de pasantía cumple con los objetivos planteados y convierte a Tuguia.de en servicio disponible a los usuarios 24 horas durante los 7 de la semanas, en cualquier lugar simplemente usando un teléfono celular Android, lo que representa una avance que puede llegar a marcar la diferencia entre varias empresas ofreciendo el mismo servicio.

\section{Resultados Generales y Estado Actual}

El proyecto se logró completar en un 100\% de lo que estaba establecido en el alcance incluso dispone de funcionalidades que aún no se encuentra disponibles en el sitio de web. A pesar de que cuando se visualiza la información de un local en la página web de Tuguia.de se muestra su ubicación sobre el mapa, esto no ocurre cuando se observa un local en la aplicación móvil. Esta funcionalidad quedó fuera del alcance de las pasantía debido a que dio prioridad al trabajo realizado con Drupal y Apache Solr que permitió realizar búsquedas más especificas dado varios parámetros. 

Este tipo de búsqueda aun no está disponible en el portal web de Tuguia.de, sin embargo gracias a las configuraciones, módulos y extensiones desarrolladas durante la pasantía, se disponen todos los mecanismos necesarios para la integración de esta funcionalidad en el sitio web. Este trabajo será realizado por el equipo de desarrollo responsable a corto plazo, agregando así un valor adicional gracias a las pasantías.


Si bien una de las ventajas mas importantes de las metodologías basadas en pruebas, es que garantizan la elaboración de productos de calidad, la versión desarrollada se encuentra en fase beta y si bien es bastante estable, aún debe ser sometida a una serie pruebas mas extensas antes de considerar realizar su despliegue.

\section{Recomendaciones}

Estas recomendaciones se realizan para mejorar el proceso de desarrollo y futuras mejoras en la aplicación móvil de Tuguia.de. 

\begin{itemize}
\item No se realizaron pruebas exhaustivas en el rendimiento de la versión final, por lo que
éstas deben ser realizadas cuando antes de que la aplicación sea lanzada al mercado.
\item Es necesario integrar a la vista de cada local su ubicación en el mapa, para hacer esto existen varias alternativas, entre las cuales resalta Google Maps, ya que mediante la integración de un conjunto de bibliotecas disponibles en el sitio oficial, se puede acceder de manera muy sencilla a su funcionalidad desde el entorno Android. Ademas esta solución ofrece un valor agregado al brindar la capacidad de mostrar en el mapa la forma de llegar desde la ubicación hasta el local.
\item Se podría mejorar la experiencia de búsqueda a través de la integración de un mecanismo para filtrar los resultados, no solo por ubicación sino por los atributos asociados a un local. Para un usuario no es suficiente sólo saber cuales son los locales que existen en un determinado lugar, también sería útil saber por ejemplo cuales tienen estacionamiento y aceptan tarjeta de crédito. Dado la configuración actual de Apache Solr esto es posible, sin embargo es necesario realizar un desarrollo en Drupal para hacer que esta funcionalidad este disponible a través del API y luego llevarla a la aplicación móvil. 
\item Actualmente la interfaz de la aplicación móvil, no se ajusta a la imagen corporativa de Tuguia.de, es necesario realizar diseños para cada una de las pantallas y luego llevarlos a los \textit{layouts} de la aplicación.

\end{itemize}
