\chapter{Conclusiones, Recomendaciones}

El desarrollo de las pasantías permitió conocer diferentes aspectos relacionados con el ámbito empresarial y el desarrollo de proyectos de manera efectiva en éste ambiente. Dentro aprendizaje adquirido resalta el benéfico del contacto directo y constante con la dirección general del proyecto, esto no sólo mejora la calidad del producto generado, sino que también permite aportar directamente al modelo de negocio y su crecimiento. 

Ha sido un reto constante para Empresas TD ofrecer soluciones tecnológicas e innovadoras orientadas a mejorar la calidad del comercio electrónico en Venezuela. Este proyecto de pasantía representa la primera aplicación móvil de desarrollo nativo que se realiza en la empresa, por lo que significó la apertura a una nueva forma de acercamiento a sus clientes y la garantía de mantenerse al día con las nuevas tendencias mundiales.

Las metodologías aplicadas al igual que otros procesos de desarrollo ágil dan prioridad a  satisfacer al cliente mediante la entrega continua de funcionalidad y la capacidad de que los requisitos cambien, incluso en etapas tardías del desarrollo. Adaptarse a esta estructura de trabajo tan dinámica es la única manera de vivir al día con los avances tecnológicos y las tendencias globales. Las empresas que logran llevar estos dos elementos de la mano, son las que tienen mayor éxito en nuestros días. 

Este dinamismo al que estuvo sujeto el proyecto afectó también muy positivamente la experiencia y el aprendizaje adquirido. Si bien el objetivo principal de la pasantía era el desarrollo de una aplicación Android, tambien se pudo interactuar con otras tecnologías como Drupal y Apache Solr, lo que representó la adquisición de un conocimiento valioso que no estaba planificado.

El proyecto de pasantía cumple con los objetivos planteados y convierte a Tuguia.de en servicio disponible a los usuarios 24 horas durante los 7 de la semanas, en cualquier lugar simplemente usando un teléfono celular Android, lo que representa una avance que puede llegar a marcar la diferencia entre varias empresas ofreciendo el mismo servicio.

A continuación se mencionan algunas recomendaciones que pueden enriquecer el proceso de desarrollo y producir mejoras en la aplicación móvil de Tuguia.de.
 
\begin{itemize}
\item No se realizaron pruebas exhaustivas en el rendimiento de la versión final, por lo que
éstas deben ser realizadas cuando antes de que la aplicación sea lanzada al mercado.
\item Es necesario integrar a la vista de cada local su ubicación en el mapa. Para ello existen varias alternativas, entre las cuales resalta Google Maps, ya que mediante la integración de un conjunto de bibliotecas disponibles en el sitio oficial, se puede acceder de manera muy sencilla a su funcionalidad desde el entorno Android. Ademas esta solución ofrece un valor agregado al brindar la capacidad de mostrar en el mapa la forma de llegar desde la ubicación hasta el local.
\item Se podría mejorar la experiencia de búsqueda a través de la integración de un mecanismo para filtrar los resultados, no solo por ubicación sino por los atributos asociados a un local. Para un usuario no es suficiente sólo saber cuales son los locales que existen en un determinado lugar, también sería útil saber por ejemplo cuales tienen estacionamiento y aceptan tarjeta de crédito. Dado la configuración actual de Apache Solr esto es posible, sin embargo es necesario realizar un desarrollo en Drupal para hacer que esta funcionalidad este disponible a través del API y luego llevarla a la aplicación móvil. 
\item Actualmente la interfaz de la aplicación móvil, no se ajusta a la imagen corporativa de Tuguia.de, es necesario realizar diseños para cada una de las pantallas y luego llevarlos a los \textit{layouts} de la aplicación.

\end{itemize}
