\section{Problema de satisfacción de restricciones} \label{sect:CSP}

Problema de las ocho reinas, colorear grafos, Sudoku y muchos otros rompecabezas 
de lógica, son un campo de investigación de la inteligencia artificial que los
termino generalizando como problemas de satisfacción de restricciones (CSP de 
sus siglas en inglés Constraint Solving Programming). Si se revisan estos 
problemas lo que los compone en general son un conjunto de objetos o 
características que deben satisfacer un conjunto de restricciones o 
limitaciones \cite{RN09}.

Una implementación de un CSP necesita 3 componentes:
\begin{itemize}
\item{\textbf{Variables:} Son todos los objetos que componen un problema. Por 
ejemplo en un Sudoku serian las $9x9$ casillas de enteros que lo componen.}
\item{\textbf{Dominios:} Son los dominios que tienen las variables del problema
en cuestión. Continuando con el ejemplo del Sudoku serian los enteros del 1 al
9 para cada casilla.}
\item{\textbf{Grafo de restricciones:} Es una representación del conjunto de 
restricciones que busca mostrar la relación entre variables. Para seguir con el
ejemplo los nodos del grafo representan las casillas del Sudoku y los enlaces 
entre ellas indican las reglas.}
\end{itemize}

Un detalle que tienen los CSP es que no consumen mucha memoria ya que iteran 
sobre los dominios verificando que combinación satisface el problema, aunque 
esto sea bueno para problemas ``fáciles'' (pocos elementos en el dominio) suele
tomar mucho tiempo encontrar una solución. Para atacar el problema del tiempo se
se han detectado patrones que pueden optimizar el cálculo. 

Una estrategia de optimización es saber elegir en que variable instanciar 
primero, esto se puede decidir viendo que variable tiene más restricciones 
asociadas y así se instancia la variable más restringida primero, así puede que 
se encuentre o no la solución más rápido.

Otra estrategia que se tiene es ordenar los los valores del dominio a asignar,
esto puede descartar muy rápido valores de asignaciones no validas.

Una medida más es determinar en que orden conviene calcular que se satisfagan
las restricciones, esto hoy en día es muy difícil de computar y entonces lo que
se hace es dejar del parte del usuario de la herramienta de CSP que coloque las
restricciones en un orden conveniente para acelerar el cálculo.
