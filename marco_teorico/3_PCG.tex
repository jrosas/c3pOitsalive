\section{Generación procedimental de contenido} \label{sect:PCG}

La creación de contenido aleatorio de forma procedimental es una forma de generar
contenido en la que se le especifica en un algoritmo paso a paso que hacer 
usando un factor aletorio o pseudo-aleatorio para ir crontruyendo la estructura
\cite{PCG}.

La mayor parte de los estudios realizados en esta área ha sido dedicada al 
desarrollo de videojuegos ya que en general esta técnica es bastante rápida y
puede ser usada en tiempo real, y eso es justamente lo que se necesita en ellos.
Algunos de los estudios realizados son aquellos que buscan la generación de 
escenarios de juegos, como el realizado por Kate Compton y Michael Mateas 
\cite{CM06}. Otro ejemplo fue a partir del estudio realizado a los jugadores en
el que se media su diversión contra la diversión que les proporcionaba, gracias 
a esto se desarrollo una inteligencia artificial que ajustaba su dificultad en
el juego \cite{HC04}.
