\section{Aspectos Tecnológicos} \label{sect:aspectos_tecnologicos}

En esta sección se presentan las tecnologías y herramientas utilizadas en el desarrollo de la pasantía, asi como tambien algunos términos asociados que son de obligatoria referencia para facilitar al lector la comprensión del contenido subsiguiente. Es necesario hacer una separación entre las tecnologías y herramientas utilizadas para el desarrollo de la aplicación móvil y las asociadas a \textit{Tuguia.de} con las que se interactuó en el transcurso de la pasantía.

\subsection{Desarrollo de la Aplicación Móvil} \label{subsect:desarrollo_movil}

\subsubsection{\textit{Teléfono Inteligente}}

Los teléfonos inteligentes o \textit{smartphones} (nombre en ingles), son teléfonos celulares que ademas de poseer las características habituales de un celular común, permiten acceder a una series de aplicaciones integradas, navegar en Internet, tomar fotografías, grabar vídeos, entre otras funcionalidades que lo asemejan  a una computadora portátil móvil \cite{PCM}.

El objetivo del desarrollo es llevar a \textit{Tuguia.de} a estos dispositivos.

\subsubsection{Android}

Android es un sitema operativo basado en \textit{Linux}, diseñado en principio para dispositivos móviles con pantalla táctil, es de código abierto y es liberado bajo la licencia Apache; una licencia bastante flexible, que permite que cualquier desarrollador realice aplicaciones y las integre al sistema sin dificultad.

Las aplicaciones nativas se pueden desarrollar de dos maneras, usando el SDK (Software Development Kit) y escribiendo la aplicación en lenguaje \textit{java}. El SDK es un paquete proporcionado directamente Google que encapsula todas las librerías y herramientas de desarrollo necesarias para crear, probar, y depurar las aplicaciones \cite{ASDK}. La segunda alternativa es usar el NDK (Native Development Kit) también provisto por Google que permite mediante el uso de lenguajes como \textit{C} y \textit{C++} interactuar directamente con el sistema operativo. Esta ultima alternativa solo es recomendada para situaciones especificas donde la aplicación en cuestión realiza cálculos intensivos sobre el \textit{CPU} y debe emplearse sólo cuando el SDK no provee de la funcionalidad deseada\cite{ANDK}.

\subsubsection{Arquitectura del Sistema Android}


asdf