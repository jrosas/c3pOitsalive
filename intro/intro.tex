\chapter*{Introducción}

\pdfbookmark[0]{Introducción}{introducion} % Sets a PDF bookmark for the dedication

En la sociedad actual se hace imprescindible que las empresas posean una pagina web; el mundo que hoy conocemos se basa en Internet y es en éste donde la gente busca las respuestas a sus inquietudes cotidianas. Actualmente en el mundo existen casi el doble de \textit{smartphones} que computadores personales \cite{DGT}. Según los resultados obtenidos en una encuesta llevada a cabo por Google en conjunto con IPSOS OTX mediaCT al final del 2010, aparte de hacer y recibir llamadas, las actividades realizadas con mayor frecuencia en los teléfonos inteligentes son: navegar en Internet, el uso de motores de búsquedas y en tercer lugar el uso de aplicaciones \cite{TMM}. Para las empresas, ya no sólo es necesario estar en la web, deben estar al alcance de los teléfonos inteligentes.

Los \textit{smartphones} han llegado a ser parte de la vida de las personas y son dispositivos mucho mas personales que una computadora portátil o de escritorio, estos teléfonos han penetrado hasta el punto que muchos usuarios raramente se separan de sus equipos móviles durante el día.

Tuguia.de es una empresa joven que pretende ser la principal guía de comercios y locales de Venezuela, su objetivo no solo es brindar información actualizada de los locales, sino que también exponer la experiencia que tuvieron los usuarios al visitar estos sitios. Tomando en cuenta esto se hace importante para Tuguia.de convertirse en un servicio disponible a los usuarios 24 horas durante los 7 días de la semana, accesible desde cualquier momento y lugar por lo que se hace necesario el desarrollo de un mecanismo para que los usuarios puedan acceder a la información del portal desde sus celulares.
%HABLAR DE TUFgUIA

Conscientes de la influencia que pueda tener el éxito de la versión móvil de Tuguia.de, se decide emprender un desarrollo que permita llevar la aplicación a los celulares inteligentes.  Sin embargo para lograr esto existen varias alternativas a considerar; aplicaciones nativas, aplicaciones híbridas o multiplataforma y paginas web diseñadas específicamente celulares.

Las aplicaciones nativas, llevan este nombre porque son desarrollos implementados en el lenguaje nativo del dispositivo. Gracias a esto, sacan el mejor rendimiento posible del \textit{hardware} del móvil, así que tienen un alto desempeño y permiten acceder a las diferentes características presentes en los teléfonos inteligentes con mayor facilidad. Sin embargo, el hecho de necesitar un desarrollo en un lenguaje específico, distinto para cada plataforma eleva los costos y el tiempo de desarrollo. 

Las aplicaciones híbridas permiten escribir código en un solo lenguaje para después exportarlo a código nativo y de esta manera obtener una aplicación que funcione en las múltiples plataformas con un solo desarrollo. Es de notar que estas aplicaciones tienen un bajo desempeño si las comparamos con las nativas, ademas en estos entornos multiplataforma se dificulta la tarea de acceder a los diferentes componentes del equipo móvil con efectividad. 

Las páginas web de diseño específico para equipos móviles son sitios web que pueden cambiar la forma en la que muestran su contenido de acuerdo al equipo en el que se esta visualizando y de esta manera aprovechar al máximo las dimensiones de las pantallas de los dispositivos. Así se consigue que con un solo desarrollo se obtiene un producto que puede operar en tanto en las computadoras como en equipos móviles. Al igual que las otras paginas web éstas se ejecutan en un navegador, por lo que se torna bastante complejo acceder a las características propias de los teléfonos inteligentes de hoy en día. 

Luego de evaluar estas alternativas, la junta directiva de Tuguia.de decide que desarrollar una aplicación nativa para su proyecto, a pesar de los costos elevados y los altos tiempos de desarrollo desean que la aplicación sea lo mas extensible posible, no obstante, es necesario determinar cuál sistema operativo móvil es el más adecuado para iniciar las versiones móviles de Tuguia.de

Según un estudio realizado para finales del 2012 en los Estados Unidos,\cite{NTD} para el tercer trimestre de ese año la participación de Android en el mercado de teléfonos inteligentes era del 52\%. Se hace notable la superioridad de Android sobre los demás sistemas operativos, por lo que es el sistema escogido para iniciar el desarrollo móvil de Tuguia.de, pensado en un futuro a corto plazo expandirse al iOS de Apple.

Actualmente Tuguia.de cuenta con funcionalidades tales como búsqueda de locales dadas sus ubicación y/o una serie de taxonomías asociadas a éstos, además siempre que se muestra información de un local se debe mostrar su ubicación en el mapa, dirección, fotografías asociadas e información de contacto. Un aspecto importante para la visión de negocio de Tuguia.de es la capacidad de que los usuarios puedan realizar comentarios y establecer una puntuación de uno a cinco  sobre los locales, esto con el objetivo de brindar no solo la información básica de un establecimiento, sino también priorizar los locales de acuerdo a la opinión de los usuarios. Adicionalmente, Tuguia.de prevé ofrecer la posibilidad a los dueños de cada local de gestionar la información de su comercio y campañas de publicidad en el sitio.

El Proyecto de pasantía consiste en llevar las funcionalidades presentes a una aplicación móvil para el sistema operativo Android. Gracias a los elementos propios de los teléfonos inteligentes como geolocalización, se pretende agregar nuevas funcionalidades y así realizar búsquedas usando la posición actual del equipo, de esta manera brindar información más precisa a los usuarios de Tuguia.de.

Para exportar la funcionalidad de Tuguia.de se desarrolló un API (\textit{Aplication Programming Interface}), que permite gestionar el contenido de la página a través de él. La aplicación móvil debe usar este API como materia prima para realizar sus operaciones y de esta forma mantener una comunicación constante con la aplicación web de Tuguia.de logrando así ofrecer contenido siempre actualizado.

Se espera que la aplicación móvil, pueda operar efectivamente en las diferentes versiones activas del sistema operativo Android, al menos en las que concentran la mayor cantidad de usuarios, además debe tener un uso mínimo de la red de datos y poseer una interfaz sencilla y fácil de usar.

Por estas razones se plantea como objetivo general del proyecto de pasantía desarrollar una aplicación móvil basada en el sistema operativo AndroidOS para el portal Tuguia.de, que permita realizar búsquedas avanzadas sobre los locales, mostrar la información detallada de un local, realizar comentarios y que posea un manejo básico de usuarios.

Los objetivos específicos que se desean alcanzar con la ejecución de este proyecto, son los siguientes:

 \begin{itemize}
\item Estudiar la arquitectura de la plataforma de desarrollo de Android, entornos de pruebas y demás tecnologías asociadas.
\item Diseñar y desarrollar un mecanismo de comunicación con el API de Tuguia.de.
\item Realizar los ajustes y extensiones necesarias sobre el API de Tuguia.de para garantizar el funcionamiento deseado.
\item Diseñar e implementar un modelo de datos que permita gestionar los diferentes nodos de información presentes en Tuguia.de dentro la aplicación móvil. 
\item Desarrollar un módulo para la aplicación móvil que permita realizar búsquedas avanzadas, usando palabras claves y una ubicación, que puede ser las coordenadas geográficas o el nombre de un determinado lugar.
\item Diseñar y desarrollar un componente que permita visualizar los locales de Tuguia.de en la aplicación móvil.  
\item Diseñar e implementar la funcionalidad que permita subir comentarios asociados a un local de Tuguia.de.
\item Implementar diferentes mecanismos que permitan que la aplicación realice el menor uso posible de la red de datos.

\end{itemize}

El presente proyecto pretende generar una versión beta de la aplicación móvil de Tuguia.de en la plataforma Android; de manera que esta sea estable y se encuentre en un estado de completitud suficiente para ser aprobada por la directiva de la empresa y pasar a una fase de prueba mas extensa pre-despliegue. 