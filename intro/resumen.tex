\setcounter{page}{4}
\pdfbookmark[0]{Resumen}{resumen} % Sets a PDF bookmark for the dedication
\begin{center}
	{\bf Resumen}
\end{center}	

Empresas TD es una empresa dedicada a ofrecer soluciones tecnológicas e innovadoras orientadas a mejorar la calidad del e-commerce en Venezuela y a brindar la oportunidad de darse a conocer a los diferentes comercios y locales de las principales ciudades del país en este medio.

Bajo este precepto Empresas TD crea el portal Tuguia.De con el objetivo de convertirse en la 
principal guía de locales y comercios del país. Dado el crecimiento de Tuguia.De como proyecto, la empresa decide llevar esta idea a las plataformas móviles comenzando con el desarrollo de una aplicación nativa para el sistema operativo AndroidOs. 

Este informe describe las actividades realizadas durante el desarrollo del proyecto de pasantía larga, que consistió en la elaboración de una aplicación móvil para el portal Tuguia.de que permita llevar la iniciativa de este portal a un mayor número de usuarios
de una forma sencilla, permitiendo así que éstos puedan acceder a los diferentes servicios prestados por esta guía desde cualquier lugar usando su teléfono celular Android.

El desarrollo de esta aplicación móvil, se realizó siguiendo el proceso iterativo e incremental de la metodología ágil Scrum, y los principios del desarrollo guiado por 
pruebas y comportamientos (\textit{Test-driven development y behavior-driven development}); 
fue realizada con el lenguaje de programación java, haciendo uso del entorno de trabajo 
dispuesto para Android (\textit{Android SDK}) y SQLite como manejador de base de datos de la aplicación en el dispositivo.

Como resultado de este proyecto de pasantía se realizó el analisís, diseño e implementación de la aplicación móvil que cubre los casos de usos y requerimientos planteados en el alcance inicial. Adicionalmente se realizó la configuración de un servidor de ApacheSolr para mejorar
el resultado de las búsquedas y diversas mejoras funcionales al API (\textit{Aplication Programming Interface}) que permite acceder a las funcionalidades del portal Tuguia.De siendo 
este la materia prima de la aplicación móvil.

